\documentclass{sig-alt-release2}
\usepackage{graphicx}
\usepackage{url}
\usepackage{textcomp}
\usepackage{times}
\usepackage{color}

\newcommand{\todo}[1]{}
% \newcommand{\todo}[1]{\footnote{{\bf TODO:} #1}}
\newcommand{\Intel}{Intel\textsuperscript{\textregistered}}

\newcommand{\idea}[1]{{\color{blue} IDEA: #1}\\}
\newcommand{\studyresult}[1]{{\color{red} STUDY RESULT?: #1}\\}

\begin{document}
%
\title{Connecting Documents to Ideas}

\numberofauthors{3}

\author{
\alignauthor Beth Trushkowsky\\
       \affaddr{Computer Science Division}\\
       \affaddr{University of California at Berkeley}\\
       \email{trush@berkeley.edu}
\alignauthor Rob Ennals\\
       \affaddr{Intel Research}\\
       \affaddr{2150 Shattuck Avenue}\\
       \affaddr{Penthouse Suite}\\
       \affaddr{Berkeley, CA 94704, USA}\\
       \email{robert.ennals@intel.com}
}

\sloppy 

\maketitle

\begin{abstract}

Much of the information contained on the web consists of opinions, arguments, blogs, articles, and other such unstructured natural language text. Each such document contains one or more statements or ideas that might be of interest to readers.

We propose a new approach to finding, organizing, and sharing ideas contained in web pages. In this approach, a user marks points of interest 

\end{abstract}

\section{Introduction}

\idea{Much interesting data on the web is unstructured text}

\idea{These documents consist of ideas}

\idea{Think Link connects documents to the ideas they express}

\idea{This paper is itself marked up in think link - with URL}

\idea{If the ideas in a document are known then this allows interesting queries}


\section{Why People Annotate}

\idea{People already annotate documents, in the form of blogs, comments etc}

\idea{People already organize data, using text files, Google Notepad, OneNote, etc}

\idea{People already share information, by mailing links, posting stories on facebook, sending messages on twitter, etc}

\idea{People already want to see what each other are interested in - using Facebook, Twitter, Friendfeed, etc}

\studyresult{Our users used these tools}

\studyresult{Users want to put forward their opinions}

\studyresult{People want to be seen as a source of useful information}

\studyresult{People want to organize the information they find}

\studyresult{Our users said they would use our tool}

\studyresult{The best way to find answers is social - by learning from friends who understand the topic}


\section{Organizing Knowledge}

\idea{Like Wikipedia without the articles}

\idea{Organize ideas with Points, snippets, and topics}

\idea{A point can support or oppose another point}

\idea{A point can be exactly the same or exactly opposite}

\idea{Important to have wiki features to keep things in check}

\studyresult{People found it useful to organize ideas}

\studyresult{People found the topic approach worked well}

\studyresult{People were able to easily find and reuse existing points and topics}


\section{Using this Knowledge}

\idea{Show me the points that other people disagree with}

\idea{Show me the points that other people found interesting}

\idea{Show me the points that my friends found interesting}

\idea{Show me the articles that make interesting points I haven't heard before}

\idea{Show me when new snippets appear on points I'm interested in}

\idea{Show me when new points appear on topics I'm interested in}


\studyresult{People liked to be able to monitor interesting new ideas on topics they were interested in}

\studyresult{People liked the other features too}


\section{Creating Knowledge}

\section{Sharing Knowledge}

Why do users g

\section{Using Knowledge}

\section{Querying Data}

\section{Related Work}

\subsection{Web Annotation Systems}

\subsection{Semantic Linking}

\subsection{User Contributed Links}

Memex allowed users to describe their own paths between pages, rather than requiring the page authors to modify them.

Everything2 creates soft-links between documents based on the browsing patterns of users.

ENQUIRE contains bi-directional links that show who links to a document. TrackBack does the same.

\subsection{Automated Reasoning}

\subsection{Argumentation Graphs}

\section{Evaluation}

\section{Conclusions}

\end{document}