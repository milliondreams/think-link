
Unlike most prior forms 


One of the really transformative properties of the web is that it allows anyone to publish whatever they want without having to go through a higher authority that must approve everything that is written. Older mediums such as print, radio, and television, have inherent costs of publication, which led us to limit the amount that was published, and have all publication gated through a small set of publishers, distributers, and channels~\cite{shirky?}. With it's much lower publication costs, the internet has removed the need for 

constraints of the medium, most information people obtain through 

A consequence of this is that users are exposed to much more information that has not been vetted by an organization with a known reputation or agenda. Due to the constraints of the media, all information transmitted through print, radio, or television has had to be approved in some way by one of a small set of publishers, 

With print media,


The downside of this is that, unlike print media, in which most information one encounters has been approved by a publisher with a known reputation, much of the information on the web is provided by sources that users know little about. When a user reads information on a web page, they often have little idea of the trustworthiness, competence, or biases of the author.


The downside of this is that, unlike print media, in which most information one encounters has been filtered through a publisher with a known reputation, much of the information on the web has been written by user

A consequence of this is that much of the information that one encounters on the web has not been certified by an authority that a user trusts, and so it becomes much harder for a user to decide whether they should trust the information they are reading.

\idea{Of course, old media wasn't entirely trustworthy either}

However the old way of doing things had certain advantages. 

One frustrating consequence of this is that 

 that much of the information on the web comes from sources that users do not know whether they can trust. 

When the available media is filtered through a small number of publishers, 

 


We currently use a combination of crowdsourcing from users, and mining web sites such as Snopes and Politifact that already maintain lists of Disputed claims. An alternative approach would be to automatically detect conflicts between statements on the web, or detect sentences that look like they are criticisi 



Like Dispute Finder, several other services highlight content that users should be suspicious of. 

WikiTrust~\cite{Adler2008a} highlights pass

Several authors have built tools that help people know when they should trust information on Wikipedia. Wikipedia\footnote{http://wikipedia.org}. WikiTrust~\cite{Adler2008a} highlights passages on wikipedia that have been written recently or written by untrusted editors. Wiki Dashboard~\cite{Kittur2008} creates a visualization of the edit history of a Wikipedia article that lets a user see how contentious it is. Wiki Scanner\footnote{http://wikiscanner.virgil.gr} finds cases where a Wikipedia edit has been made by someone with a conflict of interest. 





\todo{Make sure we stay in sync}

\todo{Talk about Skewz and Newstrust?}

\todo{Talk about WikiNews}

NewsTrust: users rank articles they read based on fairness, accuracy, context, and sourcing. But it's a lot of work. Shows a story in a frame with a toolbar above it. Not practical to apply to all news. People just read too much.

\todo{Cite previous work in topic detection and tracking on news stories?}

\todo{We should talk to NewsTrust and Skewz}

Skewz - shows you conservative and liberal stories - but why read the conservative one?
Feeds in stories and users drag them to their political bisas.

NewsCube~\cite{Park2009} and MediaCloud\footnote{http://mediacloud.org} use statistics to help readers avoid media bias. NewsCube finds articles on the same topic that have different biases. MediaCloud shows which different words different news sources associate with the same topic. Unlike Dispute Finder, these tools find contrasting pages about broad topics, rather than trying to find particular claims that are disputed or snippets that make disputed claims.



Several authors have created tools that help users know when to trust information on 
% 
% 
% NewsCube~\cite{Park2009}.
% Wikipedia fixes Vandalism~\cite{Viegas2004}.
% Trustworthiness~\cite{Gil2006}.
% WikiTrust~\cite{Adler2008}. Wiki Dashboard~\cite{Kittur2008}.
% Wiki Scanner\footnote{http://wikiscanner.virgil.gr}.
% SourceWatch\footnote{http://sourcewatch.org}, Snopes\footnote{http://snopes.com}, FactCheck\footnote{http://factcheck.org}.
